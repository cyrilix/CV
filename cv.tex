\documentclass[11pt, oneside, a4paper, french]{article}

\usepackage{times}
\usepackage[T1]{fontenc}
\usepackage[utf8]{inputenc}

\usepackage[french]{babel}
\usepackage{ulem} %double soulignement
\usepackage{fancybox}
\usepackage{graphicx}
\usepackage[A4]{vmargin}
%\usepackage[francais]{layout}
\usepackage[pdftex,colorlinks=true,
  pdftitle={CV Cyrille NOFFICIAL},
  pdfsubject={Architecte logiciel/Devops},
  pdfauthor={Cyrille Nofficial},
  pdfkeywords={JAVA Python Git Gerrit},
  pdfstartview=FitV,
  linkcolor=black,
  citecolor=black,
  urlcolor=blue]{hyperref}
\usepackage{lastpage}
\usepackage{fancyhdr}

% Marges
\setlength{\headsep}{0pt}
\setlength{\footskip}{5pt}
\setlength{\oddsidemargin}{2cm}
\setlength{\textwidth}{500pt}
\setlength{\marginparwidth}{0pt}
\setlength{\marginparsep}{0pt}

%\setmarginsrb{gauche}{haut}{droite}{bas}{ht_tete}{dist_tete_text}{ht_pied}{bas_page_bas_pied}
\setmarginsrb{2.0cm}{2cm}{1.5cm}{2cm}{0cm}{0cm}{1cm}{1cm}


%Entêtes et pied de page
\renewcommand{\headrulewidth}{0pt}% suppression du trait 
\cfoot{}
\rfoot{\thepage{}/\pageref{LastPage}}


\begin{document}
\pagestyle{fancy}


% PHOTO
\hspace{12cm}{\vspace{2cm}{ \framebox{\includegraphics[height=3cm]{photo.jpg}}}}
      
\vspace{-5cm} 

%========== Adresse ==========
\parbox[l]{9cm}{
%  \begin{flushleft}
  NOFFICIAL cyrille\\
  n\'e le 21/05/1981 \`a Lorient\\
  
  31, rue Saint Amand\\
  75015 Paris\\
  %%T\'el: 06 68 63 64 48\\
  %%\textit{E-mail:} \texttt{cyrille\_nofficial@yahoo.fr}
}  


%========== Entete ==========
\vspace{1cm}
\begin{center}
\Ovalbox{
  \begin{minipage}{12cm}
    \begin{center}
      \LARGE{Architecte Logiciel / Devops}
    \end{center}
  \end{minipage}
}
\end{center}
\vspace{1cm}

\textbf{
  \Large{
    \uuline{Comp\'etences:}
  }
}\\
\begin{description}
\item[OS:] Linux/UNIX (Debian, Centos, Solaris).
\item[Langages:] Java, Python, Bash
\item[Int\'egration continue] Git, Gerrit, Jenkins/Hudson, Sonar, Nexus
\item[J2EE:] Tomcat, Glassfish, JEE, JAX-WS, JAX-RS, Spring, Logback
\item[NoSQL:] MongoDB, Redis
\item[Message Queuing:] AMQP, RabbitMQ
\item[Base de donn\'ees:] SQL, PostgreSQL/PostGIS, Oracle, Mysql
\item[Outils de d\'eveloppement:] Maven, Ant, Git, SVN, CVS, Vim, Redmine
\item[Web:] Nginx, Lighttpd, Apache
\item[Auto H\'ebergement:] postfix, dovecot, roudcube, owncloud
\item[Autres:] openssl, VirtualBox, LVM, SoapUI, doxygen, graphviz, \LaTeX, ssh
\end{description}



 \vspace{1cm}
%% \centerline{
%%   \rule{8cm}{0.01cm}
%% }
%% \vspace{0.5cm}


\textbf{
  \Large{
    \uuline{Exp\'erience Professionnelle:}
  }
}\\

\begin{description}
\setlength{\itemsep}{25pt}

%% Architecte logiciel %%
\item[depuis Novembre 2011:]
    Achitecte logiciel
    \begin{description}
        \item[T\^ache:]
            \begin{description}
                \item Adaption de l'usine logiciel à l'offshorisation (externalisation partielle des développements en Russie (Novosibirsk))
                \item Mise en place de l'architecture des syst\'emes techniques.
                \item D\'eveloppement des outils tranverses (framework de monitoring fonctionnel)
                \item Mise en place et administration de l'usine logicielle: jenkins, sonar, ...
                \item Mise en place et administration d'outils de monitoring temps r\'eel de la production
            \end{description}
    \end{description}

%% Bougues développeur %%
\item[Novembre 2009 - Novembre 2011:] Bouygues Telecom\\
    D\'eveloppement de webservices.
    \begin{description}
        \item[T\^ache:] TODO       
    \end{description}

%% Bouygues prestataire%
\item[depuis Ao\^ut 2008:] Neotilus (Groupe Degetel)\\
  Bouygues Telecom Fixe - D\'eveloppement d'un webservice J2EE.
  \begin{description}
  \item[T\^ache:] Impl\'ementation d'un webservice permettant la souscription d'abonnement depuis les boutiques du R\'eseau Club Bouygues Telecom.
  \item[Technologies:] Serveur J2EE SJSAS 9.1/Glassfish v2, Oracle 9.4, Eclipse, Netbeans, Ant, JAXB, JAX-WS, JAX-RPC, git, cvs, doxygen, Junit, XSLT, Jmx
  \end{description}

%% LiveRobot
\item[Juin 2008 - Juillet 2008 :] Neotilus (Groupe Degetel)\\
  Projet LiveRobot: d\'eveloppement d'un prototype d'application permettant la prise de contrôle à distance d'un robot jouet. 
  \begin{description}
  \item[T\^ache:] mis en place d'une application serveur charg\'ee de diffuser la vid\'eo du robot vers les clients connect\'es et les instructions de d\'eplacement vers le robot.
  \item[Technologies:] Tomcat, Ibatis, Mysql, Eclipse, Junit 
  \end{description}

%% Webraska %%
\item[F\'evrier 2008 - Mai 2008:]  Neotilus (Groupe Degetel)\\
Webraska, Maisons-Laffitte - D\'eveloppement d'un moteur de pois dynamique permettant l'int\'egration et la mise \`a jour de pois.
  \begin{description}
  \item[T\^ache:] D\'eveloppement de la couche stockage reposant sur les fonctionnalit\'es spatiales de Mysql et int\'egration d'un webservice d\'eploy\'e en frontal.
  \item[Technologies:] Tomcat, Mysql, JAX-WS (Apache CXF), Eclipse, Junit
  \end{description}


%% Magic %%
\item[Septembre 2007 - F\'evrier 2008 :]  Neotilus (Groupe Degetel)\\
  Projet Magic (Ratp): recherche d'itin\'eraires multi-modal (transport en commun/pi\'eton) et multi-provider (Ratp, Webraska).
  \begin{description}
  \item[T\^ache:] D\'eveloppement d'une API de calculs d'itin\'eraires pi\'eton s'appuyant sur les web services de diff\'erents fournisseurs (RATP, Webraska, PTV, Yahoo) int\'egrant les fonctionnalit\'es suivantes:
    \begin{description}
    \item[-] geocoding et reversegeocoding
    \item[-] calculs d'itin\'eraires
    \item[-] r\'ecup\'eration de cartes
    \item[-] trac\'e d'itin\'eraires et affichage de pois sur des cartes existantes (ind\'ependamment de l'origine des cartes, des itin\'eraires et des 
    \end{description}
    
  \item[Technologies:] Java, PostgreSQL/Postgis, PGrouting, ImageMagick, JTS (Java Topology Suite), Ant, Junit
  \end{description}
  
%% SNCF - GEM %%
\item[Mars 2007 - F\'evrier 2008 :] Neotilus (Groupe Degetel)\\
  SNCF - Gares en Mouvement: TMA du site web \href{http://www.gares-en-mouvement.com}{www.gares-en-mouvement.com}
  \begin{description}
  \item[T\^ache:] Maintenance de la couche pr\'esentation (Front-Office) bas\'e sur un gestionnaire de contenu Java (OpenCMS)
    
  \item[Technologies:] OpenCms, Spring, Tomcat, Apache, Eclipse, CVS, Ant, HTML, JS, JSP, Taglib, CSS, Log4j, PostgreSQL
    
    
  \end{description}

%% Sytadin %%
\item[D\'ecembre 2005 - Septembre 2007 :] Neotilus (Groupe Degetel)\\
  SISER - Sytadin: R\'ealisation du site internet d'information routi\`ere \href{http://www.sytadin.fr}{www.sytadin.fr}
  \begin{description}
  \item[T\^ache:]
    \begin{description}
    \item[-] D\'eveloppement de la couche pr\'esentation (Front et Back-Office) reposant sur un gestionnaire de contenu Java (OpenCMS)
    \item[-]Installation et configuration du serveur Linux (Debian)
    \end{description}

  \item[Technologies]  OpenCms, Spring, Tomcat, Apache, Eclipse, CVS, Ant, HTML, JavaScript, JSP, Taglib, CSS, Log4j, PostgreSQL, Junit, Debian
  \end{description}

%% Stage %%
\item[Juin 2005 - D\'ecembre 2005] France Telecom Recherche \& D\'eveloppement (Stage Master 2)\\
 Algorithmique d'un calculateur d'itin\'eraires  appliqu\'e au guidage p\'edestre.
 \begin{description}
 \item[T\^ache:] D\'eveloppement de la base de donn\'ees et du calculateur:
   \begin{description}
   \item[-] Fusion de deux bases de donn\'ees (Teleatlas, IGN)
   \item[-] Reconstruction de la topologie du graphe routier issue des donn\'ees sources 
   \item[-] Algorithme et impl\'ementation du calculateur d'itin\'eraires en Java
   \item[-] G\'en\'eration des instructions de guidage
   \item[-] G\'en\'eration des cartes sous forme vectorielle (SVG) et raster (png, jpg)
   \end{description}
 \item[Technologies:] Java, C, SVG, PostgreSQL/PostGIS, Linux (Debian), Batik, shapefile
 \end{description}
\end{description}


\vspace{1cm}

%%\newpage
\textbf{
  \Large{
    \uuline{Dipl\^omes:}
  }
}\\
\newcounter{vi} \setcounter{vi}{6} %Ecriture en chiffres romains
\begin{description}
  \item[2004-2005:] Master Paris \Roman{vi} fili\`ere Science de l'Information G\'eographique, anciennement DESS double comp\'etence Informatique Appliqu\'ee aux Sciences de laTerre
  \item[2003-2004:] DEA G\'eosciences Marines, sp\'ecialit\'e G\'eologie Structurale et S\'edimentaire (Brest)
  \item[2002-2003:] Maitr\^ise Sciences de la Terre et de l'Univers (Brest)
  \item[2001-2002:] Licence Sciences de la Terre et de l'Univers (Brest)
\end{description}

\vspace{1cm}
%
%

\textbf{
  \Large{
    \uuline{Langues vivantes:}
  }
}\\

\begin{description}
  \item[\textbf{Anglais:}] lu, \'ecrit
\end{description} 

\vspace{1cm}

\textbf{
  \Large{
    \uuline{Activit\'es:}
  }
}

\begin{description}
    \item[-] auto-h\'ebergement (serveur mail, owncloud, ...)
    \item[-] contributeur sur le projet OpenStreetMap.org (r\'egion de Lorient)
\end{description}

\vspace{1cm}

\textbf{
  \Large{
    \uuline{Divers:}
  }
}\\

\noindent
Permis de conduire cat\'egorie B


\end{document}
