\documentclass[11pt, a4paper, french]{article}

\usepackage{times}
\usepackage[T1]{fontenc}
\usepackage[french]{babel}
\usepackage{ulem} %double soulignement
\usepackage{fancybox}
\usepackage{graphicx}
\usepackage[A4]{vmargin}

\begin{document}


\noindent
% Marges
\setlength{\footskip}{5pt}
\setlength{\oddsidemargin}{1cm}
%\setlength{\textwidth}{450pt}
%\setlength{\marginparwidth}{0pt}
%\setlength{\marginparsep}{0pt}
%\setmarginsrb{gauche}{haut}{droite}{bas}{ht_tete}{dist_tete_text}{ht_pied}{bas_page_bas_pied}
\setmarginsrb{2.5cm}{2cm}{0.0cm}{0cm}{0cm}{0cm}{0cm}{0cm}

\unitlength=1cm
%\layout
\pagestyle{empty}
% PHOTO
%\begin{picture}(3,2)(-12,3.5)
  % \centering
\hspace{12cm}
       {\vspace{1cm}{ \framebox{\includegraphics[height=3cm]{photo.png}}}}
%\end{picture}
\vspace{-5cm} 



%========== Adresse ==========
\parbox[l]{9cm}{
%  \begin{flushleft}
  NOFFICIAL cyrille\\
  n\'e le 21/05/81 \`a Lorient\\
  
  10, rue Dombasle\\
  75015 Paris\\
    
  T\'el: 06 22 95 15 08\\
  \textit{E-mail:} \texttt{cyrille\_nofficial@yahoo.fr}
}  


%========== Entete ==========
\vspace{1cm}
\begin{center}
\Ovalbox{
  \begin{minipage}{12cm}
    \begin{center}
      \LARGE{Ing\'enieur d'\'etude Java/J2EE}
    \end{center}
  \end{minipage}
}
\end{center}
\vspace{1cm}
%
%
%
\setlength{\textwidth}{450pt}

\textbf{
  \Large{
    \uuline{Comp\'etences:}
  }
}\\
\begin{description}
\item[OS:] Linux/UNIX (Debian, Fedora, Suse, Solaris).
\item[Langages:] Java, Python, Bash
\item[J2EE:] Tomcat, SJSAS/Glassfish, J2EE4, J2EE5,
JAX-WS, Spring, Tomcat, Struts, Ant, Log4j, Hibernate, Ibatis
\item[Base de donn\'ees:] SQL, Pl/Pgsql, PostgreSQL/PostGIS, Mysql
\item[Outils de d\'eveloppement:] Ant, Eclipse, Netbeans, Git, SVN, CVS, Emacs, Trac
\item[Packaging:] Ant, Debian (sources et binaires)
\item[Web:] Apache, HTML, CSS, JSP
\item[Autre:] xsl, doxygen, graphviz, \LaTeX, ssh
\end{description}



\vspace{0.5cm}
\centerline{
  \rule{8cm}{0.01cm}
}
\vspace{0.5cm}


\textbf{
  \Large{
    \uuline{Exp\'erience Professionnelle:}
  }
}\\

\begin{description}
\item[Ao\^ut 2008 - :] Bouygues Telecom Fai - D\'eveloppement d'un webservice J2EE
  \begin{description}
\item[T\^ache: ] Impl\'ementation d'un webservice permettant la souscription d'abonnement depuis les boutiques du r\'eseau
\item[-] g\'en\'eration de contrat sous forme de PDF
  \item[-] Serveur J2EE SJSAS 8.2 et 9.1
  \item[-] Base de donn\'ees Oracle 9.4
  \item[-] Netbeans, Ant,
  \item[-] technologies: JAXB, JAX-WS, JAX-RPC, git, cvs, doxygen, Junit
  \end{description}

\item[Juin 2008 - Juillet 2008 :] Projet LiveRobot (OrangeLab)\\
  D\'eveloppement d'une servlet de contr\^ole \`a distance (commande et vid\'eo) 
  \begin{description}
  \item[-] Tomcat
  \item[-] Eclipse, Junit, Ibatis

  \end{description}

%% Webraska %%
\item[F\'evrier 2008 - Mai 2008:] Webraska, Maisons-Laffitte - D\'eveloppement d'un moteur de pois dynamique permettant l'int\'egration et la mise \`a jour de pois\\
  \begin{description}
  \item[T\^ache:]
    \begin{description}
      \item[-] D\'eveloppement de la couche stockage bas\'ee sur Mysql
      \item[-] D\'eveloppement du webservice d\'eploy\'e en frontal bas\'e sur JAX-WS/Apache CXF
    \end{description}
  \item[Technologies:] Tomcat, Mysql, JAX-WS (Apache CXF), Eclipse, Junit
  \end{description}


%% Magic %%
\item[Septembre 2007 - F\'evrier 2008 :] Projet Magic (Ratp)\\
  Recherche d'itin\'eraires multi-modal (transport en commun/pi\'eton) et multi-provider (Ratp, Webraska)
  \begin{description}
  \item[T\^ache:]
    \begin{description}
    \item[-] D\'eveloppement d'une API de calculs d'itin\'eraires pi\'eton/ratp s'appuyant sur les web services de diff\'erents fournisseurs (RATP, Webraska, PTV, Yahoo). Cette API int\`egre les fonctionnalit\'es suivantes:
      \begin{description}
        \item[.] geocoding et reversegeocoding
        \item[.] calculs d'itin\'eraires
        \item[.] r\'ecup\'eration de cartes
        \item[.] trac\'e d'itin\'eraires et affichage de pois sur des cartes existantes(ind\'ependamment de l'origine des cartes, des itin\'eraires et des pois) 
      \end{description}
    \end{description}
    
  \item[Technologies:] Java, Postgis/PGrouting, ImageMagick, JTS(Java Topology Suite), PostgreSQL/Postgis, Junit
  \end{description}
  
%% SNCF - GEM %%
\item[Mars 2007 - F\'evrier 2008 :] SNCF - Gares en Mouvement\\
  TMA du site web www.gares-en-mouvement.com
  \begin{description}
  \item[T\^ache:] Maintenance de la couche pr\'esentation (Front-Office) bas\'e sur un gestionnaire de contenu Java (OpenCMS)
    
  \item[Technologies:] OpenCms, Spring, Tomcat, Apache, Eclipse, CVS, Ant, HTML, JS, JSP, Taglib, CSS, Log4j, PostgreSQL
    
    
  \end{description}

%% Sytadin %%
\item[D\'ecembre 2005 - Septembre 2007 :] SISER - Sytadin\\
  R\'ealisation du site internet d'information routi\`ere ``www.sytadin.fr''
  \begin{description}
  \item[T\^ache:]
    \begin{description}
    \item[-] D\'eveloppement de la couche pr\'esentation (Front et Back-Office) reposant sur un gestionnaire de contenu Java (OpenCMS)
    \item[-]Installation et configuration de la plateforme Linux (Debian)
    \item[-]Tests
    \item[-] D\'eploiement
    \end{description}

  \item[Technologies]  OpenCms, Spring, Tomcat, Apache, Eclipse, CVS, Ant, HTML, JavaScript, JSP, Taglib, CSS, Log4j, PostgreSQL, Junit, Debian
  \end{description}

%% Stage %%
\item[Juin 2005 - D\'ecembre 2005] France Telecom Recherche \& D\'eveloppement (Stage Master 2)\\
 Algorithmique d'un calculateur d'itin\'eraires  appliqu\'e au guidage p\'edestre
 \begin{description}
 \item[T\^ache:]
   \begin{description}
   \item[-] Conception et impl\'ementation d'une base de donn\'ees g\'eographiques utilisant PostgreSQL et sa composante g\'eograhique PostGIS
   \item[-] Fusion de deux bases de donn\'ees (Teleatlas, IGN)
   \item[-] Reconstruction de la topologie du graphe routier issue des donn\'ees sources 
   \item[-] G\'en\'eration de donn\'ees manquantes \`a partir des donn\'ees de la base (g\'eome\'etrie des rues...)
   \item[-] Algorithme et impl\'ementation du calculateur d'itin\'eraires en Java
   \item[-] G\'en\'eration des instructions textuels n\'ecessaire au guidage
   \item[-] G\'en\'eration des cartes sous forme vectorielle (SVG) et raster (png, jpg)
   \end{description}
 \item[Technologies:] Java, C, SVG, PostgreSQL/PostGIS, Linux (Debian), Batik, shapefile
 \end{description}
\end{description}


\centerline{
  \rule{8cm}{0.01cm}
}
\vspace{0.5cm}
%
%
%
\textbf{
  \Large{
    \uuline{Dipl\^omes:}
  }
}\\
\newcounter{vi} \setcounter{vi}{6} %Ecriture en chiffres romains
\begin{description}
  \item[2004-2005:] Master Paris \Roman{vi} fili\`ere Science de l'Information G\'eographique, anciennement DESS double comp\'etence Informatique Appliqu\'ee aux Sciences de laTerre
  \item[2003-2004:] DEA G\'eosciences Marines, sp\'ecialit\'e G\'eologie Structurale et S\'edimentaire (Brest)
  \item[2002-2003:] Maitr\^ise Sciences de la Terre et de l'Univers (Brest)
  \item[2001-2002:] Licence Sciences de la Terre et de l'Univers (Brest)
\end{description}

\vspace{0.5cm}
\centerline{
  \rule{8cm}{0.01cm}
}
\vspace{0.5cm}
%
%

\textbf{
  \Large{
    \uuline{Langues vivantes:}
  }
}\\

\begin{description}
  \item[\textbf{Anglais:}] lu(B), parl\'e(P), \'ecrit(P)
  \item[\textbf{Espagnol:}] scolaire
\end{description} 

\vspace{0.5cm}
\centerline{
  \rule{8cm}{0.01cm}
}
\vspace{0.5cm}

\textbf{
  \Large{
    \uuline{Activit\'es:}
  }
}\\

VTT

\vspace{0.5cm}
\centerline{
  \rule{8cm}{0.01cm}
}
\vspace{0.5cm}

\textbf{
  \Large{
    \uuline{Divers:}
  }
}\\

\noindent
Permis de conduire cat\'egorie B


\end{document}
