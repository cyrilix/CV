\documentclass[11pt, oneside, a4paper, french]{article}

\usepackage{times}
\usepackage[T1]{fontenc}
\usepackage[utf8]{inputenc}

\usepackage[french]{babel}
\usepackage{ulem} %double soulignement
\usepackage{fancybox}
\usepackage{graphicx}
\usepackage[A4]{vmargin}
%\usepackage[francais]{layout}
\usepackage[pdftex,colorlinks=true,
  pdftitle={CV Cyrille NOFFICIAL},
  pdfsubject={Ingénieur d’étude Java/J2EE},
  pdfauthor={Cyrille Nofficial},
  pdfkeywords={JAVA J2EE spring hibernate},
  pdfstartview=FitV,
  linkcolor=black,
  citecolor=black,
  urlcolor=blue]{hyperref}
\usepackage{lastpage}
\usepackage{fancyhdr}

% Marges
\setlength{\headsep}{0pt}
\setlength{\footskip}{5pt}
\setlength{\oddsidemargin}{2cm}
\setlength{\textwidth}{500pt}
\setlength{\marginparwidth}{0pt}
\setlength{\marginparsep}{0pt}

%\setmarginsrb{gauche}{haut}{droite}{bas}{ht_tete}{dist_tete_text}{ht_pied}{bas_page_bas_pied}
\setmarginsrb{2.0cm}{2cm}{1.5cm}{2cm}{0cm}{0cm}{1cm}{1cm}


%Entêtes et pied de page
\renewcommand{\headrulewidth}{0pt}% suppression du trait 
\cfoot{}
\rfoot{\thepage{}/\pageref{LastPage}}


\begin{document}
\pagestyle{fancy}


% PHOTO
\hspace{12cm}{\vspace{2cm}{ \framebox{\includegraphics[height=3cm]{photo.jpg}}}}
      
\vspace{-5cm} 

%========== Adresse ==========
\parbox[l]{9cm}{
%  \begin{flushleft}
  NOFFICIAL cyrille\\
  n\'e le 21/05/81 \`a Lorient\\
  
  10, rue Dombasle\\
  75015 Paris\\
  %%T\'el: 06 22 95 15 08\\
  %%\textit{E-mail:} \texttt{cyrille\_nofficial@yahoo.fr}
}  


%========== Entete ==========
\vspace{1cm}
\begin{center}
\Ovalbox{
  \begin{minipage}{12cm}
    \begin{center}
      \LARGE{Ing\'enieur d'\'etude Java/J2EE}
    \end{center}
  \end{minipage}
}
\end{center}
\vspace{1cm}

\textbf{
  \Large{
    \uuline{Comp\'etences:}
  }
}\\
\begin{description}
\item[OS:] Linux/UNIX (Debian, Fedora, Suse, Solaris).
\item[Langages:] Java, Python, Bash
\item[J2EE:] Tomcat, SJSAS/Glassfish, J2EE4, J2EE5,
JAX-WS, Spring, Struts, Ant, Log4j, Hibernate, Ibatis
\item[Base de donn\'ees:] SQL, Pl/Pgsql, PostgreSQL/PostGIS, Oracle, Mysql
\item[Outils de d\'eveloppement:] Ant, Eclipse, Netbeans, Git, SVN, CVS, Emacs, Trac, Mantis
\item[Packaging:] Ant, Debian
\item[Web:] Apache, HTML, CSS, JSP
\item[Autres:] XSLT, doxygen, graphviz, \LaTeX, ssh
\end{description}



 \vspace{1cm}
%% \centerline{
%%   \rule{8cm}{0.01cm}
%% }
%% \vspace{0.5cm}


\textbf{
  \Large{
    \uuline{Exp\'erience Professionnelle:}
  }
}\\

\begin{description}
\setlength{\itemsep}{25pt}

\item[depuis Ao\^ut 2008:] Neotilus (Groupe Degetel)\\
  Bouygues Telecom Fai - D\'eveloppement d'un webservice J2EE.
  \begin{description}
  \item[T\^ache:] Impl\'ementation d'un webservice permettant la souscription d'abonnement depuis les boutiques du r\'eseau Club Bouygues Telecom.
  \item[Technologies:] Serveur J2EE SJSAS 8.2 et 9.1, Oracle 9.4, Netbeans, Ant,
 JAXB, JAX-WS, JAX-RPC, git, cvs, doxygen, Junit, XSLT
  \end{description}

%% LiveRobot
\item[Juin 2008 - Juillet 2008 :] Neotilus (Groupe Degetel)\\
  Projet LiveRobot: d\'eveloppement d'un prototype d'application permettant la prise de contrôle à distance d'un robot jouet. 
  \begin{description}
  \item[T\^ache:] mis en place d'une application serveur chargée de diffuser la vidéo du robot vers les clients connectés et les instructions de déplacement vers le robot.
  \item[Technologies:] Tomcat, Ibatis, Mysql, Eclipse, Junit 
  \end{description}

%% Webraska %%
\item[F\'evrier 2008 - Mai 2008:]  Neotilus (Groupe Degetel)\\
Webraska, Maisons-Laffitte - D\'eveloppement d'un moteur de pois dynamique permettant l'int\'egration et la mise \`a jour de pois.
  \begin{description}
  \item[T\^ache:] D\'eveloppement de la couche stockage reposant sur les fonctionnalités spatiales de Mysql et intégration d'un webservice d\'eploy\'e en frontal.
  \item[Technologies:] Tomcat, Mysql, JAX-WS (Apache CXF), Eclipse, Junit
  \end{description}


%% Magic %%
\item[Septembre 2007 - F\'evrier 2008 :]  Neotilus (Groupe Degetel)\\
  Projet Magic (Ratp): recherche d'itin\'eraires multi-modal (transport en commun/pi\'eton) et multi-provider (Ratp, Webraska).
  \begin{description}
  \item[T\^ache:] D\'eveloppement d'une API de calculs d'itin\'eraires pi\'eton s'appuyant sur les web services de diff\'erents fournisseurs (RATP, Webraska, PTV, Yahoo) intégrant les fonctionnalit\'es suivantes:
    \begin{description}
    \item[-] geocoding et reversegeocoding
    \item[-] calculs d'itin\'eraires
    \item[-] r\'ecup\'eration de cartes
    \item[-] trac\'e d'itin\'eraires et affichage de pois sur des cartes existantes (ind\'ependamment de l'origine des cartes, des itin\'eraires et des 
    \end{description}
    
  \item[Technologies:] Java, PostgreSQL/Postgis, PGrouting, ImageMagick, JTS (Java Topology Suite), Ant, Junit
  \end{description}
  
%% SNCF - GEM %%
\item[Mars 2007 - F\'evrier 2008 :] Neotilus (Groupe Degetel)\\
  SNCF - Gares en Mouvement: TMA du site web \href{http://www.gares-en-mouvement.com}{www.gares-en-mouvement.com}
  \begin{description}
  \item[T\^ache:] Maintenance de la couche pr\'esentation (Front-Office) bas\'e sur un gestionnaire de contenu Java (OpenCMS)
    
  \item[Technologies:] OpenCms, Spring, Tomcat, Apache, Eclipse, CVS, Ant, HTML, JS, JSP, Taglib, CSS, Log4j, PostgreSQL
    
    
  \end{description}

%% Sytadin %%
\item[D\'ecembre 2005 - Septembre 2007 :] Neotilus (Groupe Degetel)\\
  SISER - Sytadin: R\'ealisation du site internet d'information routi\`ere \href{http://www.sytadin.fr}{www.sytadin.fr}
  \begin{description}
  \item[T\^ache:]
    \begin{description}
    \item[-] D\'eveloppement de la couche pr\'esentation (Front et Back-Office) reposant sur un gestionnaire de contenu Java (OpenCMS)
    \item[-]Installation et configuration du serveur Linux (Debian)
    \end{description}

  \item[Technologies]  OpenCms, Spring, Tomcat, Apache, Eclipse, CVS, Ant, HTML, JavaScript, JSP, Taglib, CSS, Log4j, PostgreSQL, Junit, Debian
  \end{description}

%% Stage %%
\item[Juin 2005 - D\'ecembre 2005] France Telecom Recherche \& D\'eveloppement (Stage Master 2)\\
 Algorithmique d'un calculateur d'itin\'eraires  appliqu\'e au guidage p\'edestre.
 \begin{description}
 \item[T\^ache:] D\'eveloppement de la base de donn\'ees et du calculateur:
   \begin{description}
   \item[-] Fusion de deux bases de donn\'ees (Teleatlas, IGN)
   \item[-] Reconstruction de la topologie du graphe routier issue des donn\'ees sources 
   \item[-] Algorithme et impl\'ementation du calculateur d'itin\'eraires en Java
   \item[-] G\'en\'eration des instructions de guidage
   \item[-] G\'en\'eration des cartes sous forme vectorielle (SVG) et raster (png, jpg)
   \end{description}
 \item[Technologies:] Java, C, SVG, PostgreSQL/PostGIS, Linux (Debian), Batik, shapefile
 \end{description}
\end{description}


\newpage
\textbf{
  \Large{
    \uuline{Dipl\^omes:}
  }
}\\
\newcounter{vi} \setcounter{vi}{6} %Ecriture en chiffres romains
\begin{description}
  \item[2004-2005:] Master Paris \Roman{vi} fili\`ere Science de l'Information G\'eographique, anciennement DESS double comp\'etence Informatique Appliqu\'ee aux Sciences de laTerre
  \item[2003-2004:] DEA G\'eosciences Marines, sp\'ecialit\'e G\'eologie Structurale et S\'edimentaire (Brest)
  \item[2002-2003:] Maitr\^ise Sciences de la Terre et de l'Univers (Brest)
  \item[2001-2002:] Licence Sciences de la Terre et de l'Univers (Brest)
\end{description}

\vspace{1cm}
%
%

\textbf{
  \Large{
    \uuline{Langues vivantes:}
  }
}\\

\begin{description}
  \item[\textbf{Anglais:}] lu, \'ecrit
\end{description} 

\vspace{1cm}

\textbf{
  \Large{
    \uuline{Activit\'es:}
  }
}

\begin{description}
  \item[-] contributeur sur le projet OpenStreetMap.org
  \item[-] VTT
\end{description}

\vspace{1cm}

\textbf{
  \Large{
    \uuline{Divers:}
  }
}\\

\noindent
Permis de conduire cat\'egorie B


\end{document}
